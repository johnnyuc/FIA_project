% Document settings
%--------------------------------------------------------------------
\documentclass{article}
\usepackage{comment}
\usepackage{graphicx} % Required for inserting images
\usepackage{fancyhdr} % Required for custom headers
\usepackage{makeidx} % Required for indexes
\usepackage{lastpage} % Required for last page
\usepackage{indentfirst} % Required to indent the first paragraph

% Page Settings
%--------------------------------------------------------------------
\pagestyle{fancy} % Using fancy
\usepackage[a4paper, total={6in, 8in}]{geometry} % Document sizes
\usepackage[T1]{fontenc} % Font
\usepackage{lmodern} % Font
\renewcommand{\baselinestretch}{1.2} % Line spacing
\setcounter{page}{-1} % Set starting page
\makeindex % Index

% Settings for Fancy Headers and Footers
%-------------------------------------------------------------------
\fancyhf{} % Clean header/footer settings
\fancyhead[L]{\leftmark} % Header text
\renewcommand{\sectionmark}[1]{\markboth{#1}{}} % Remove header section number
\setlength{\headheight}{12.80502pt} % Header font height
\fancyfoot[R]{\thepage\ de \pageref{LastPage}} % Footer

% Settings for paragraphs
%-------------------------------------------------------------------
\setlength{\parindent}{20pt} % Paragraph indent
\setlength{\parskip}{5pt} % Space between paragraphs

% Document cover
%-------------------------------------------------------------------
\begin{document}
\sffamily % Define font

% Logo
\begin{figure}
    \centering
    \includegraphics[width=1\linewidth]{images/logo.jpg}
\end{figure}

% Title
\title{Fundamentos de Inteligência Artificial}

% Authors
\author{
\begin{tabular}{ll}
    Johnny Fernandes & 2021190668 \\
    Miguel Leopoldo & 2021225940 \\
\end{tabular}
}

% Date
\date{Março 2024}
\maketitle

% Footer page number settings
\thispagestyle{empty} % Remove page number
\newpage

% Index page
%-------------------------------------------------------------------
\renewcommand*\contentsname{Índice}
\tableofcontents

\thispagestyle{empty} % Remove page number
\newpage

% 1
%-------------------------------------------------------------------
\section{Introdução}

\newpage

% 2
%-------------------------------------------------------------------
\section{Modelação}
O comportamento dos fantasmas do jogo é pautado por um conjunto perceções e
ações fornecidas no arquivo do projeto e no enunciado do presente trabalho. Desta lista consta uma perceção que define o vetor de direções disponíveis: protected List<Vector2> getAvailableDirections(Node node);.
Esta perceção fornece-nos uma lista de vetores de direção bidimensionais, e que
estão disponíveis para a tomada de decisão do caminho a seguir, tendo a seguinte ordem: Norte, Sul, Oeste, Este.

Esta orientação é primordial na definição dos comportamentos dos
fantasmas bem como dos seus sistemas de produção. A ordem mencionada é também
demonstrada pela movimentação base dos fantasmas caso o seu comportamento ainda
não tenha sido implementado.
Assim, convém definir em traços gerais o comportamento de cada fantasma no jogo,
para que possamos posteriormente demonstrar o seu sistema de produções.

\begin{itemize}
    \item \textbf{Blinky} {Fantasma Vermelho}
    
    Este fantasma possui o comportamento mais ofensivo, sendo que se
movimenta na direção do jogador (Pac-Man) pelo tabuleiro, procurando a cada
tomada de decisão – na atribuição de uma nova direção – o caminho mais rápido e
direto que é possível obter (cuja distância é mínima). Para isso usa o método das
distâncias de Manhattan, usando a implementação de Vector3.Distance para
medir a distância em todas as direções no momento do trigger entre
getPacmanPosition e currentPosition (ambas perceções) e, para todas as
possíveis distâncias calculadas com as direções disponibilizadas pela por
getAvailableDirections, toma-se a direção cuja distância é menor para atingir
o jogador, usando a ação setDirection.
\end{itemize}

\begin{itemize}
    \item \textbf{Pinky} {Fantasma Rosa}
    
    À semelhança do fantasma Blinky, este também tem um comportamento
ofensivo, procurando atingir o jogador. A forma como o Pinky se movimenta no
tabuleiro é através do cálculo de uma distância entre o fantasma e a posição do
quarto tile à sua frente. Para isso calculamos a posição do Pac-Man com a perceção
getPacmanPosition onde lhe somamos quatro vezes o vetor direção nesse
instante através da perceção getPacmanDirection e, com base nessa localização
definida, o Pinky movimentar-se-á da mesma forma que no caso do Blinky,
procurando o caminho mais eficiente usando as distâncias de Manhattan e
movendo-se com a ação setDirection.
\end{itemize}

\begin{itemize}
    \item \textbf{Inky} {Fantasma Azul}
    
    Este fantasma tem um comportamento aleatório, fazendo apenas uso da
perceção getAvailableDirections e da ação setDirection, tomando uma
escolha aleatória de direção através do uso da função Random.Range(0, count)
em que count é o número de direções disponíveis. Assim, este fantasma vagueia
pelo tabuleiro sob a mera expectativa de poder atingir o jogador, sendo também por
isso facilmente evitável durante o jogo. Movimenta-se, à semelhança dos casos
anteriores, com a ação setDirection.
\end{itemize}

\begin{itemize}
    \item \textbf{Clyde} {Fantasma Laranja}
    
    Por fim, o comportamento deste (quarto e último) fantasma é afastando-se dos
seus congéneres tanto quanto possível. Neste caso o fantasma faz uso das
perceções getAvailableDirections e getClosestGhostPosition tomando a
escolha da direção que maximize a distância em relação ao fantasma mais
próximo. A forma como calcula esta distância é, mais uma vez, através do uso de
Vector3.Distance, movendo-se com a ação setDirection.
\end{itemize}

>  Quinto comportamento (ainda não implementado)

% 3

\section{Sistemas de produções}

\begin{itemize}
    \item \textbf{Blinky} {Fantasma Vermelho}

    1. Se está no modo chase: Persegue o Pac-Man usando a menor distância entre
currentPosition+getAvailableDirections[i] e getPacmanPosition;
    2. Se está no modo Frightened: Foge do Pac-Man usando a maior distância entre
currentPosition+getAvailableDirections[i] e getPacmanPosition;
\end{itemize}
 
\begin{itemize}
    \item \textbf{Pinky} {Fantasma Rosa}

    1. Se está no modo chase: Persegue o quarto tile à frente getPacmanDirection
usando para isso o cálculo da menor distância entre
currentPosition+getAvailableDirections[i] e getPacmanPosition
+ 4*getPacmanDirection;
    2. Se está no modo Frightened: Foge do Pac-Man usando a maior distância entre
currentPosition+getAvailableDirections[i] e getPacmanPosition;
\end{itemize}

\begin{itemize}
    \item \textbf{Inky} {Fantasma Azul}

    1. Se está no modo chase: Movimenta-se aleatoriamente usando a função
Random na escolha do vetor de direção obtido por getAvailableDirections;
    2. Se está no modo Frightened: Foge do Pac-Man usando a maior distância entre
currentPosition+getAvailableDirections[i] e getPacmanPosition;
\end{itemize}

\begin{itemize}
    \item \textbf{Clyde} {Fantasma Laranja}

    1. Se está no modo chase: Foge do fantasma mais próximo, calculando a maior
distância entre currentPosition+getAvailableDirections[i] e
getClosestGhostPosition;
    2. Se está no modo Frightened: Foge do Pac-Man usando a maior distância entre
currentPosition+getAvailableDirections[i] e getPacmanPosition;
\end{itemize}

\begin{comment}
[Exemplo do Paiva]
Sistema de produções do Blinky:
    1.    PACN ^ !PN ^ !DS -> AN
    2.    PACN ^ !PE ^ !DO -> AE
    3.    PACN ^ !PO ^ !DE-> AO
    4.    PACS ^ !PS ^!DN -> AS
    5.    PACS ^ !PE ^ !DO -> AE
    6.    PACS ^!PO ^!DE -> AO
    7.    PACE ^! PE ^ !DO -> AE
    8.    PACE ^ !PN ^!DS -> AN
    9.    PACE ^ !PS ^ !DN -> AS
    10.  PACO ^ !PO ^ !DE -> AO
    11.  PACO ^ !PN ^!DS -> AN
    12.  PACO ^ !PS  ^ !DN -> AS
\end{comment}

\end{document}
